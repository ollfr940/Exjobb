\chapter{Preprocessing of data}
\label{sec:Preprocessing of data}

A big amount of gray scale images containing different kinds of code will be available. For each image the corresponding ground truth will also be available. One part of the images will be used as training data and the rest will be used as test data. The idea is to divide each image into blocks of same size. The amount of training data
\begin{center}

\begin{math}
A_{m,n} =
 \begin{pmatrix}
  a_{1,1} & a_{1,2} & \cdots & a_{1,n} \\
  a_{2,1} & a_{2,2} & \cdots & a_{2,n} \\
  \vdots  & \vdots  & \ddots & \vdots  \\
  a_{m,1} & a_{m,2} & \cdots & a_{m,n}
 \end{pmatrix}
\end{math}
\end{center}
will then be the number of blocks in each image times the number of training images. The blocks can either overlap each other or just lay next each other. Overlapped blocks will lead to higher accuracy but more data have to be processed. Each data will consist of a feature vector:

Each feature will in some way describes the corresponding block. The only information that is available is the intensities of the pixels in each block; consequently all features will always, in some way, be based on the pixel values.


% Local Variables:
% TeX-master: "main.tex"
% End:

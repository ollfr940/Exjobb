\chapter{Features used for detecting barcodes}
\label{sec:Features used for detecting barcodes}

\section{Standard deviation}
\label{sec:Standard deviation}
For the first step in the cascade a good method is to simply compute the standard deviation of each block. Blocks which contain code will have a high standard deviation; hence all data with standard deviation under a certain threshold can be discarded. In this step the amount of data will be decreased a lot. 

\section{Structure tensor}
\label{sec:Structure tensor}
In the next steps one might consider to compute the gradients in the blocks. This can be done by convolving the images with a sobel filter. The gradient image can then be used in several ways. One way is to calculate the eigenvalues of the structure tensor for each block and then use these to estimate the
structure inside the block. This can be a good way to distinguish between 1D-code and 2D-code.  In a block containing 1D-code the gradient will only vary in one direction, this means it will have an i1D- structure. However in a block containing 2D-code the variation will be fairly equal in both directions. The structure tensor can also be used to calculate Harris-corners, which can be used as a feature

\section{FAST corner detection}
\label{sec:FAST corner detection}

\section{Distance map}
\label{sec:Distance map}

\section{Local binary pattern}
\label{sec:Local binary pattern}
One feature that might be considered is the so called Local Binary Pattern, which is described in [3]. The basic idea is to compute a binary code in every pixel, based on the difference of the intensity between the pixel and the surrounding pixels, illustrated in figure 2. The binary code will then be transformed to a decimal scalar value. If a 3x3 neighborhood is used there will be 256 different possible values. For each block a histogram will be calculated for all these values. Every bin in the histogram will then be used as a feature. If there is a bin for every possible value, there will be 256 features.
% Local Variables:
% TeX-master: "main.tex"
% End:

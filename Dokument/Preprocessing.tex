\chapter{Data processing}
\label{sec:Data processing}

\section{The dataset}
\label{The dataset}
The dataset consists of a big amount of gray scale images containing 1D- and 2D-codes of different sizes and orientation. For each image the corresponding ground truth are also available. The images are first of all divided into one training dataset and one testing dataset. Each image are then divided into tiles of the same size. 
\begin{center}
	$x = [x_1,...,x_N]$
\end{center}
With corresponding ground truth:
\begin{center}
	$y = [y_1,...,y_N]$
\end{center}
The tiles can either overlap each other or just lay next to each other. For each tile a number of different features are calculated, i.e. each tile consists of a feature vector:
\begin{center}
\begin{math}
	x_i =  
	\begin{pmatrix}
		 f_1 \\ \vdots \\ f_M
	\end{pmatrix}
\end{math}
\end{center}  
The features are in some way based on the pixel values in the tile.

During testing each data will get a prediction. These can have four different states based on the relation between the prediction and the ground truth:
\begin{itemize}
	\item true positive
	\item false positive
	\item true negative
	\item false negative
\end{itemize}

\section{Preprocessing}
\label{Preprocessing}
Some of the features used in the system gives better result with some preprocessing. The reason for this is that the codes in some of the images have less contrast than others. Here a Laplace filter is applied to the images to make them a bit sharper. However some of the features gives better result without any preprocessing, therefore the filtered images are only used for calculation of some of the features.

\section{Postprocessing}
\label{Postprocessing}
At the end of the cascade when all the test data have been predicted there are some postprocessing before obtaining the final result. The reason for this is to remove some possible true false classifications. The false classifications are often distinctive since the true true classifications usually lay in clusters. Here a filter is applied which removes isolated true classifications. After that some morphological operations are done to fill some possible holes in the areas where the codes are.
% Local Variables:
% TeX-master: "main.tex"
% End:

\chapter{Introduction}
\label{sec:introduction}
Machine learning is a technology that can be utilized in many ways within the field of automatic manufacturing and logistics. It can involve automatic inspection, recognition and quality control. 

This project has been carried out at the company SICKIVP in Linköping. The objective has been to investigate the use of machine learning for detection and recognition in images. In these applications a suitable method is to use supervised learning. This means that labeled training data is used to train a classifier. This classifier will then be used to classify unlabeled test data. Both during training and testing the classifier needs features extracted from the data. What kind of features that are used is one of the most critical aspects for a good classifier.

\section{Objective}
This thesis is part of a pilot study which is currently under way at SICKIVP in Linköping. The study comprises different ways to utilize machine learning in cameras used in manufacturing. The methods that are currently used for detection and recognition involves different use of template matching. It is of interest to investigate if methods involving machine learning gives a better result.

In this thesis the objective has been to investigate the use of machine learning for: 
\begin{itemize}
	\item Detection of barcodes, both 1D- and 2D-codes
	\item Optical character recognition (OCR)
\end{itemize}

Both systems has been implemented in C++ together with OpenCV. The supervised machine learning methods that are of most interest for these application is:
\begin{itemize}
	\item Discrete AdaBoost
	\item Random forest
\end{itemize}

For barecode detection one of the objectives has been to find some suitable features. Since the system needs to be relatively fast to work in a real situation one challenge is to use features that are less computational but still effective. These features can also be combined in different ways which will affect speed of the system significantly. The objective has also been to train the classifiers with different methods and compare the results. Here the methods AdaBoost and Random forest has been tried out.

For OCR only Random forest has been used for training since it is much more suitable in this case. For OCR two different features have been tried out. The first one involves comparison of pixel values in point pairs which are randomly distributed in the image. This has originally been used for face detection which is presented in \citep{Nenad}. The second one is based on the Haar-features presented in \citep{Viola:2010}. This method has been used in an earlier project regarding OCR at SICKIVP. One thing that is of interest is to compared these two.
 
\section{Limitations}
Regarding the use of machine learning for barcodes it should be emphasized that it only includes detection. The objective is not to read the codes in any way, only to detect where they are in the images.
 
There exist many different types of 2D barcodes and many of them look rather similar to each other. However in this project only Maxicodes have been used.

It is difficult to give any number regarding the demanded speed of the systems. It is desired to have methods which are as fast as possible, so this is something that should be included in the evaluation. 


% Local Variables:
% TeX-master: "main.tex"
% End:

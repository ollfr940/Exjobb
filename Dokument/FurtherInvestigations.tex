\chapter{Further investigations}
\label{sec:Further investigations}
There are a lot of different 2D-codes that are currently used. The system has so far only been tested on Maxicodes. Since other kinds of 2D-codes have similar structure they will presumably give a very similar result. Maxicodes should be more difficult to detect then most other types of 2D-codes since it consists of small circular areas. Most other types of 2D-codes, e.g. QR-codes consists instead of small squares, this should make them even more distinguishable then Maxicodes.

The system could also be tested on a more complicated data set. If one would use a data set that contains more details in the background the system would probably detect a lot more false tiles. However many of the data sets that are available look very similar to the one that has been used.

To speed up the system even further one possibility could be to train a cascade for the local binary pattern alone. Since the LBP has 256 features it is likely that some of these features are more common than others for 2D-codes. If the system is trained to use these features in a certain order it would probably speed up the system. However since the LBP features have been used in the last step of the cascade and the amount of data already has been reduced a lot it is uncertain if this will make much impact.

Postprocessing 

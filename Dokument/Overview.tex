\chapter{System overview}
\label{sec:System overview}

\section{Borcode detection overview}
\label{sec:Barcode detection overview}
The system for detection of barcodes is basically divided in two parts. The data consist of a big amount of images which contain 1D- and 2D-codes. Each image has a corresponding ground truth where the parts of the image containing codes has been labeled as "true" and the rest as "false". The first part of the system involves the training using machine learning which produces a classifier. It consists of the following steps.

\begin{itemize}
	\item Preprocessing of images.
	\item Divide images and ground truth into tiles.
	\item Calculate features for each tile.
	\item Use machine learning to train the dataset.
\end{itemize}

The objective of the second part of the system is to predict unlabeled data in real time. This is done with the trained classifiers and a cascade. It involves the following steps:

\begin{itemize}
	\item Preprocessing of image.
	\item In each step of the cascade calculate some specific features.
	\item Use the corresponding classifier for each feature to reduce the amount of data between each step in the cascade.
	\item Postprocessing of the output.
\end{itemize}
An overview of the system can be seen in \ref{system}.

%Flow chart
\begin{figure}[H]
\centering
\tikzstyle{largeblock} = [rectangle, draw, fill=blue!20, 
    text width=10em, text centered, rounded corners, minimum height=5em]
\tikzstyle{block} = [rectangle, draw, fill=blue!20, 
    text width=5em, text centered, rounded corners, minimum height=4em]
\tikzstyle{line} = [draw, -latex']
\tikzstyle{cloud} = [draw, ellipse,fill=red!20, node distance=3cm,
    minimum height=2em]
    
\begin{tikzpicture}[node distance = 2.5cm, auto]
    % Place nodes
    \node [cloud] (training) {labeled training data};
    \node [largeblock, below of=training] (machine) {training using machine learning};
    \node [block, below of=machine] (classifier) {classifier};
    \node [cloud, right of=training, node distance=4cm] (test) {test data};
    \node [block, below of=test, node distance=5cm] (cascade) {cascade};
    \node [block, below of=cascade] (prediction) {prediction};
    
    
    % Draw edges
    \path [line] (training) -- (machine);
    \path [line] (machine) -- (classifier);
    \path [line] (classifier) -- (cascade);
    \path [line] (test) -- (cascade);
    \path [line] (cascade) -- (prediction);

\end{tikzpicture}
\caption{Overview of the system}
\label{system}
\end{figure}

\section{OCR overview}
\label{sec:OCR overview}

%Flow chart
\begin{figure}[H]
\centering
\tikzstyle{largeblock} = [rectangle, draw, fill=blue!20, 
    text width=10em, text centered, rounded corners, minimum height=5em]
\tikzstyle{block} = [rectangle, draw, fill=blue!20, 
    text width=6em, text centered, rounded corners, minimum height=4em]
\tikzstyle{line} = [draw, -latex']
\tikzstyle{cloud} = [draw, ellipse,fill=red!20, node distance=3cm,
    minimum height=2em]
 
\begin{tikzpicture}[node distance = 2.5cm, auto]
    % Place nodes
    \node [cloud] (trainingImage) {image for training};
    \node [largeblock, below of=trainingImage] (trainingData) {produce training data};
    \node [largeblock, below of=trainingData] (machine) {training using Random forest};
    \node [block, below of=machine] (classifier) {classification forest};
    \node [cloud, right of=trainingImage, node distance=4cm] (test) {test data};
    \node [block, below of=test, node distance=7.5cm] (cascade) {cascade};
    \node [block, below of=cascade] (prediction) {prediction};
    
    
    % Draw edges
    \path [line] (trainingImage) -- (trainingData);
    \path [line] (trainingData) -- (machine);
    \path [line] (machine) -- (classifier);
    \path [line] (classifier) -- (cascade);
    \path [line] (test) -- (cascade);
    \path [line] (cascade) -- (prediction);

\end{tikzpicture}
\caption{Overview of the system for OCR}
\label{system OCR}
\end{figure}


% Local Variables:
% TeX-master: "main.tex"
% End:
